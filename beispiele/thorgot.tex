\documentclass{m5figur-mv}[2018/10/19]
\begin{document}

\DefData{Spieler}{Martin Väth}

\DefData{Figur}{Thorgot}
\DefData{Typ}{Zwerg (Or)}
\DefData{Grad}{3}
\DefData{Spezialisierung}{}
\DefData{St}{78}
\DefData{Gs}{91}
\DefData{Gw}{{50}{50}}% Normal bzw. in Rüstung
\DefData{Ko}{61}
\DefData{In}{55}
\DefData{Zt}{98}
\DefData{Au}{65}
\DefData{pA}{90}
\DefData{Wk}{65}
\DefData{B}{{19}{15}}% Normal bzw. in Rüstung
\DefData{GiT}{\StandardGiT}% Das ist Voreinstellung
\DefData{Raufen}{8}
\DefData{Ausdauerbonus}{10}
\DefData{Schadensbonus}{3}
\DefData{Angriffsbonus}{{1}{1}}% Normal bzw. in Rüstung
\DefData{Abwehrbonus}{{0}{0}}% Normal bzw. in Rüstung
\DefData{ResistenzBonus}{{3}{4}}
\DefData{Zauberbonus}{2}
\DefData{RK}{{KR}{3}}
\DefData{Ruestung}{Helm+Visier(PR), Hals(KR)/Bein(LR), Hören/Sehen -2}
\DefData{Sehen}{{0}{6}}
\DefData{Nachtsicht}{{2}{8}}
\DefData{Hoeren}{{0}{6}}
\DefData{Riechen}{{0}{6}}
\DefData{SechsterSinn}{{0}{6}}
\DefData{LP}{17}
\DefData{AP}{21}
\DefData{GG}{}
\DefData{SG}{1}
\DefData{Geburtsdatum}{}
\DefData{Alter}{28}
\DefData{haendig}{rechts}
\DefData{Groesse}{141}
\DefData{Gestalt}{k/n}
\DefData{Gewicht}{75}
\DefData{Stand}{Volk}
\DefData{Heimat}{}
\DefData{Glaube}{Zornal Eisenhand}
\DefData{Merkmale}{}
\DefData{Datum01}{{20.10.18}{200}{}{}}
\DefData{Datum02}{{23.12.18}{360}{20}{400}}

% Praxispunkte für Zauber:

\DefData{Beherrschen}{}
\DefData{Erkennen}{}
\DefData{Formen}{}
\DefData{Zerstoeren}{}
\DefData{Dweomer}{}
\DefData{Bewegen}{}
\DefData{Erschaffen}{}
\DefData{Veraendern}{}
\DefData{Wundertaten}{}
\DefData{Zauberlieder}{---}

\DefData{Abwehr}{{\StandardAbwehrPlus{0}}% Normal
	{\StandardAbwehrPlus{2}}}% Mit Verteidigungswaffe

\Fertigkeit1{Abwehr}[12]
\Fertigkeit1{Zaubern}[12]
\Fertigkeit1{Resistenz}[12]

\Fertigkeit1{Trinken}[6]
\Fertigkeit1{Wahrnehmung}[6]
\Fertigkeit1{Richtungssinn}[12]
\Fertigkeit1{Robustheit}[9]
\Fertigkeit1[Fallenentdecken]{Fallen entdecken}[3]
\Fertigkeit1{}
\Fertigkeit1{Meditieren}[12]
\Fertigkeit1{Erste Hilfe}[8]
\Fertigkeit1[Anfuehren]{Anführen}[8]
\Fertigkeit1{Klettern}[12]
\Fertigkeit1{Seilkunst}[12]
\Fertigkeit1{Reiten}[12]
\Fertigkeit1[Gelaendelauf]{Geländelauf}[12]

\Fertigkeit2[Stich]{Stichwaffe}[5]
\Fertigkeit2[1Schlag]{Einhandschlagwaffe}[6]
\Fertigkeit2[1Schwert]{Einhandschwert}[5]
\Fertigkeit2[2Schlag]{Zweihandschlagwaffe}[8]
\Fertigkeit2[Schild]{Schilde}[3]
\Fertigkeit2{}
\Fertigkeit2{}
\Fertigkeit2{}
\Fertigkeit2{}
\Fertigkeit2{}
\Fertigkeit2{}
\Fertigkeit2{Zwergisch}[12]
\Fertigkeit2{Albisch}[12]
\Fertigkeit2[SchreibenZwergisch]{Schreiben Zwergisch}[12]
\Fertigkeit2{Zauberschrift}[8]

\Fertig1{Trinken}{}
\Fertig1{Wahrnehmung}{}
\Fertig1{Richtungssinn}{}
\Fertig1{Robustheit}{}
\Fertig1{Fallenentdecken}{}
\Fertig1{}{}
\Fertig1{Meditieren}{Wk}
\Fertig1{Erste Hilfe}{pA}
\Fertig1{Anfuehren}{pA}
\Fertig1{Klettern}{St}
\Fertig1{Seilkunst}{Gs}
\Fertig1{Reiten}{Gw}
\Fertig1{Gelaendelauf}{Gw}

\Fertig2{Zwergisch}{In}
\Fertig2{Albisch}{In}
\Fertig2{SchreibenZwergisch}{In}
\Fertig2{Zauberschrift}{In}


% Das folgende Kommando ist ganz ähnlich zu \Fertig, aber für Waffen:

%\Waffe1[Beschreibung]{Referenz}[Modifikation]{AB/leer}[Schaden][Nah]
% Wie bei \Fertig, nur dass AB für den Angriffsbonus steht, und dass [Schaden]
% und [Nah] der Schaden bzw Nahkampfschaden ist; beides ist optional.
% Wird allerdings [Schaden] ausgelassen, muss auch [Nah] ausgelassen werden.
% Die "Waffe" Raufen kann automatisch referenziert werden.
%
% Für den Schaden kann das Macro \PlusSB{Wert} benutzt werden:
% Dies führt dazu, dass die Summe "Wert + Schadensbonus" mit einem
% davorgestellten "+" (bzw. "-" im negativen Fall) ausgegeben wird;
% falls die Summe 0 ist, wird nichts ausgegeben.
% Dadurch sind Angaben wie beispielsweise "1W6\plusSB{-1}" bei einem Dolch,
% "1W5\plusSB{}" bei einem Kurzschwert (der leere Eintrag hat den Wert 0)
% oder "1W5\plusSB{1}" bei einem Langschwert sinnvoll.
% Beachten Sie, dass die Benutzung dieses Macros nicht immer sinnvoll ist:
% Bei einer Fernwaffe beispielsweise wird der Schadensbonus ja nicht addiert.

\Waffe1[Langschwert]{1Schwert}{AB}[1W6\PlusSB{1}]
\Waffe1[Handaxt]{1Schlag}{AB}[1W6\PlusSB{0}]%[1\,PP]
\Waffe1[Dolch]{Stich}{AB}[1W6\PlusSB{-1}]
\Waffe1[mag.\ Dolch]{Stich}{AB}[1W6\PlusSB{0}]
\Waffe1[Stielhammer]{2Schlag}[2]{AB}[2W6\PlusSB{0}]
\Waffe1[großer Schild]{Schild}{2}
\Waffe2{Raufen}{}

%\Zauber1{Name}{AP}{Prozess}{Zauberdauer}{Reichweite}{Wirkungsziel}%
%       {Wirkungsbereich}{Wirkungsdauer}{Beschreibung}{Art}

\Zauber1{Wagemut}{2}{Verändern}{10\,sec}{---}{Geist}%
  {Zauberer}{2\,min}{S.~148; Angriff +2, Schaden +2, Abwehr -2}{Geste}
\Zauber1{Heiliger Zorn}{2}{Verändern}{Augenblick}{---}{Körper}%
  {Zauberer}{2\,min}{S.~144; Stärke +30 (Raufen +1, Schaden +2)}{Gedanke}
%\Zauber1{Segnen}{2}{Wund./""Veränd.}{1\,min}{Berührung}{Körper}%
% {1 Wesen}{10\,min}{S.~146; +1 Erfolgs/Widerstands "~5 Prüfwürfe}{Geste}
\Zauber1{Handauflegen}{1}{Erschaffen}{10\,sec}{Berührung}{Körper}%
  {1 Wesen}{0}{S.~143; 1W6 AP}{Geste}
\Zauber1{Heilen von Wunden}{3}{Erschaffen}{1\,min}{Berührung}{Körper}%
  {1 Wesen}{0}{S.~144; 1W6 LP+AP}{Geste}
\Zauber1{Erkennen der Aura}{1}{Erkennen}{Augenblick}{0\,m}{Geist}%
  {30\,m Kegel}{0}{S.~139}{Gedanke}
\Zauber1{Strahlender Panzer}{2}{Formen}{10\,sec}{Berührung}{Umgebung}%
  {Zauberer}{2\,min}{S.~147; Bannen v.\ Dunkelheit, RK +2}{Geste}
\Zauber1{Blutmeisterschaft}{1+1 je Wunde}{Verändern}{10\,sec}{---}{Körper}%
  {Zauberer}{60\,min}{S.~138; 1 LP weniger pro Wunde aber 1 AP}{Gedanke}


% Ebenfalls auf dem Zauberblatt stehen die Liste der magischen Gegenstände.
% Diese wird mit folgendem Kommando gefüllt:

%\Magisch{Gegenstand}{Beschreibung}

\Magisch{großes Schildamulett gegen Untote}{ABW 20}

\Output{Figurenblatt}
\Output{Abenteuerblatt}
\Output{Zauberblatt}

\end{document}
